%------------------Preambel----------------
% !BIB TS-program = biber
%
% das Papierformat zuerst, Schriftgröße 12 Punkte, Dokumentstil Artikel, Zweiseitig , Koma - Script
\documentclass[a4paper,12pt,twoside]{scrreprt}
% deutsche Silbentrennung
\usepackage[ngerman]{babel}
\selectlanguage{ngerman}
\usepackage[babel, german=quotes]{csquotes} 
% Deutsche Umlaute
\usepackage[utf8]{inputenc}
%
%\usepackage{helvet}
%  Now, where the change of font is desired, add the following command:
%\fontfamily{phv}\selectfont
% I have changed the font type from Times Roman to Helvetica of the title of an article.
%\title{\fontfamily{phv}\selectfont{\huge{\bfseries{Parallelization of DC-Analyzer}}}}
% wir wollen auf jeder Seite eine Überschrift
\pagestyle{headings}
% Für den Index \index{Schlüsselwort}
\usepackage{makeidx}
\makeindex
%\usepackage[backend=biber,style=mla]{biblatex}
%     BibLaTeX    http://sunsite.informatik.rwth-aachen.de/ftp/pub/mirror/ctan/macros/latex/contrib/biblatex/doc/biblatex.pdf
%\usepackage{natbib}
\usepackage[
backend=biber,
style=uni-wtal-ger,        % Zitierstil numeric, uni-wtal-ger http://biblatex.dominik-wassenhoven.de/biblatex-dw.shtml
isbn=false,                % ISBN nicht anzeigen, gleiches geht mit nahezu allen anderen Feldern
pagetracker=true,          % ebd. bei wiederholten Angaben (false=ausgeschaltet, page=Seite, spread=Doppelseite, true=automatisch)
maxbibnames=50,            % maximale Namen, die im Literaturverzeichnis angezeigt werden (ich wollte alle)
maxcitenames=3,            % maximale Namen, die im Text angezeigt werden, ab 4 wird u.a. nach den ersten Autor angezeigt
autocite=inline,           % regelt Aussehen für \autocite (inline=\parancite)
block=space,               % kleiner horizontaler Platz zwischen den Feldern
backref=true,              % Seiten anzeigen, auf denen die Referenz vorkommt
backrefstyle=three+,       % fasst Seiten zusammen, z.B. S. 2f, 6ff, 7-10
date=short,                % Datumsformat
]{biblatex}
%\setlength{\bibitemsep}{1em}     % Abstand zwischen den Literaturangaben
%\setlength{\bibhang}{2em}        % Einzug nach jeweils erster Zeile

% Festlegung Art der Zitierung - Havardmethode: Abkuerzung Autor + Jahr
%\bibliographystyle{myalpha} % alphadin
\bibliography{literatur}  % Bibtex-Datei wird schon in der Preambel eingebunden


%\usepackage[parfill]{parskip}    % Activate to begin paragraphs with an empty line rather than an indent
\usepackage{graphicx}
\usepackage{amssymb}
\usepackage{epstopdf}
%\usepackage{tipa}		%LaTeX für Linguisten http://homepage.ruhr-uni-bochum.de/alexander.linke-2/linguistik/LaTeX/downloads/latex_fuer_linguisten.pdf

% —– Start: Angaben zur Formatierung von Überschriften —– % 
\usepackage{titletoc} 
\titlecontents{chapter}[1.5em]{\addvspace{1pc}\bfseries}{\contentslabel{1.5em}} 
{\hspace*{-1.5em}}{\titlerule*[0.3pc]{.}\contentspage} 
\titlecontents{section}[3.7em]{}{\contentslabel{2.2em}}{} 
{\titlerule*[0.3pc]{.}\contentspage} 
\titlecontents{subsection}[6.7em]{}{\contentslabel{2.95em}}{} 
{\titlerule*[0.3pc]{.}\contentspage} 
\setcounter{tocdepth}{3} 
\setcounter{secnumdepth}{3} 
% —– Ende: Angaben zur Formatierung von Überschriften —– % 

%------------------Dokument----------------
\begin{document}
%---------------------
% Eigene Mekros
%------------------------Titelseite--------------------------
% Institut für Germanistische und Allgemeine Literaturwissenschaft der RWTH Aachen
\newcommand {\rwthtitelseite} [7] {
\begin{titlepage}
%
\noindent 			% Keine Einrückung im folgenden Absatz.
Institut für Germanistische und\\
Allgemeine Literaturwissenschaft\\
der RWTH Aachen
%
\begin{center}
%
\vfill
\vfill
\vfill
%
{\LARGE #2} \\		% Titel
#3 					% Untertitel - Beschreibung
\vfill
vorgelegt als\\
{\large #1} \\		% Thesistyp
\vfill
\vfill
%
%
{\large  #4 } \\ 	% Autorname
#6 den \today 		% Ort
%
\vfill
\vfill
%
\hfill
\noindent 			% Keine Einrückung im folgenden Absatz.
% \parbox [t][Höhe][b]{Breite} {Text}
\parbox [t][4cm][b]{60mm} 
{#4, #5\\
#7\\
#6 den \today } 	% Ort
%
%\hrulefill 		% Eine horizontale Linie.
%
\end{center}
\end{titlepage}
}
%
%
% Mehr Makros.

%---------------------
% Die Titelseite
\rwthtitelseite{Thesistyp}{Titel}{Untertitel}{Autorname}{Ort}
%---------------------
% Das Inhaltsverzeichnis
\tableofcontents
%---------------------
% Der Inhalt
%---------------------
% Kommentar
%
%
%---------------------
%
%
% Der Inhalt
%

%---------------------
% Die Titelseite
\rwthtitelseite{Thesistyp}{Titel}{Untertitel}{Autorname}{Matrikelnummer}{Ort}{Studiengang}
%---------------------
% Das Inhaltsverzeichnis
\tableofcontents
%---------------------
% Die Kapitel
%---------------------

\input{kapitel_einleitung}

% Kommentar

\chapter{Der Herr der Ringe}

Lorem ipsum dolor sit amet, consetetur sadipscing elitr, sed diam nonumy eirmod tempor invidunt ut labore et dolore magna aliquyam erat, sed diam voluptua. At vero eos et accusam \footcite[Vgl.][11\psq]{hdr} et justo duo dolores et ea rebum. Stet clita kasd gubergren, no sea takimata sanctus est \footcite[Vgl.][22\psq]{hobbit} Lorem \Footcite[Vgl.][29\psq]{hobbit} ipsum dolor sit amet. Lorem ipsum dolor sit amet, consetetur sadipscing elitr, sed diam nonumy eirmod tempor invidunt ut labore et dolore magna aliquyam erat, sed diam voluptua. At vero eos et accusam et justo duo dolores et ea rebum. Stet clita kasd gubergren, no sea takimata sanctus est Lorem ipsum dolor sit amet.

\section{Sauron}

Lorem ipsum dolor sit amet, consetetur sadipscing elitr, sed diam nonumy eirmod tempor invidunt ut labore et dolore magna aliquyam erat, sed diam voluptua. At vero eos et accusam et justo duo dolores et ea rebum. Stet clita kasd gubergren, no sea takimata
sanctus est Lorem ipsum dolor sit amet. Lorem ipsum dolor sit amet, consetetur sadipscing elitr, sed diam nonumy eirmod tempor invidunt ut labore et dolore magna aliquyam erat, sed diam voluptua. At vero eos et accusam et justo duo dolores et ea rebum. Stet clita kasd gubergren, no sea takimata sanctus est Lorem ipsum dolor sit amet.


% Kommentar

\chapter{Der Hobbit}

Lorem ipsum dolor sit amet, consetetur sadipscing elitr, sed diam nonumy eirmod tempor invidunt ut labore et dolore magna aliquyam erat, sed diam voluptua. At vero eos et accusam et justo duo dolores et ea rebum. Stet clita kasd gubergren, no sea takimata sanctus est Lorem ipsum dolor sit amet. Bilbo \index{Bilbo} Lorem ipsum dolor sit amet, consetetur sadipscing elitr, sed diam nonumy eirmod tempor invidunt ut labore et dolore magna aliquyam erat, sed diam voluptua. At vero eos et accusam et justo duo dolores et ea rebum. Stet clita kasd gubergren, no sea takimata sanctus est Lorem ipsum dolor sit amet.
\cite[Vgl.][83\psq]{hdr}:  Als Prefix können Hinweise wie vgl. oder siehe eingefügt werden, im Suffix können bspw. Seitenangaben hinterlegt werden. Dabei werden Zahlen als Suffix automatisch mit “S.” (bzw. dem sprachlichen Äquivalent) ergänzt. f. oder ff. erreicht man durch

%---------------------

% Ausgabe Bibliography
\printbibliography
%---------------------
% Der Index
\printindex
%---------------------

%---------------------
% Ausgabe Bibliography
\printbibliography
%---------------------
% Der Index
\printindex
\end{document}  